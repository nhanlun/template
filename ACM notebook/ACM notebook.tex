\documentclass[A4 paper, 12pt]{article}

\usepackage[a4paper, total={6.75in, 10in}]{geometry}
\usepackage{listings}
\usepackage{xcolor}
\usepackage[utf8]{inputenc}
\usepackage{enumitem}

\definecolor{codegreen}{rgb}{0,0.6,0}
\definecolor{codegreener}{rgb}{0,0.7,0}
\definecolor{codepurple}{rgb}{0.58,0,0.82}
\definecolor{backcolour}{rgb}{0.9,0.9,0.9}
\definecolor{codeblue}{rgb}{0.1,0.457,1} 

\lstdefinestyle{mystyle}{
    backgroundcolor=\color{backcolour},
    commentstyle=\color{codegreen},
    keywordstyle=\color{codeblue},
    numberstyle=\color{codegreener},
    stringstyle=\color{codepurple},
    basicstyle=\ttfamily\footnotesize,
    breakatwhitespace=false,         
    breaklines=true,                 
    captionpos=b,                    
    keepspaces=true,                 
    numbers=left,                    
    numbersep=5pt,                  
    showspaces=false,                
    showstringspaces=false,
    showtabs=false,                  
    tabsize=4
}

\lstset{style=mystyle}

\title{ACM Notebook team HCMUS-KMN}
\author{KMN}
\begin{document}
	\maketitle
	\newpage

	\section{Some define}
	\begin{lstlisting}[language=C++]
#include <bits/stdc++.h>
 
#define maxN
#define matrix_size 2
#define base 1000000007LL
#define eps 1e-8

#define cross(A,B) (A.x*B.y-A.y*B.x)
#define dot(A,B) (A.x*B.x+A.y*B.y)
#define ccw(A,B,C) (-(A.x*(C.y-B.y) + B.x*(A.y-C.y) + C.x*(B.y-A.y))) // positive when ccw
#define CROSS(a,b,c,d) (a*d - b*c)
\end{lstlisting}
	\section{Geometry}
	\begin{lstlisting}[language=C++]
struct line
{
	double a,b,c;
	line() {}
	line(double A,double B,double C):a(A),b(B),c(C){}
	line(Point A,Point B)
	{
    	a=A.y-B.y; b=B.x-A.x; c=-a*A.x-b*A.y;
	}
};
 
Point intersect(line AB,line CD)
{
	AB.c=-AB.c; CD.c=-CD.c;
	double D=CROSS(AB.a,AB.b,CD.a,CD.b);
	double Dx=CROSS(AB.c,AB.b,CD.c,CD.b);
	double Dy=CROSS(AB.a,AB.c,CD.a,CD.c);
	if (D==0.0) return Point(1e9,1e9);
	else return Point(Dx/D,Dy/D);
}
\end{lstlisting}
	\section{Dinic}
	\begin{lstlisting}[language=C++]
bool BFS()
{
	queue<int> q;
	for (int i=1; i<=n; i++) d[i]=0,Free[i]=true;
	q.push(s);
	d[s]=1;
	while (!q.empty())
	{
    	int u=q.front(); q.pop();
    	for (int i=0; i<DSK[u].size(); i++)
    	{
        	int v=DSK[u][i].fi;
        	if (d[v]==0 && DSK[u][i].se>f[u][v])
        	{
            	d[v]=d[u]+1;
            	q.push(v);
        	}
    	}
	}
	return d[t]!=0;
}
 
int DFS(int x,int delta)
{
	if (x==t) return delta;
	Free[x]=false;
	for (int i=0; i<DSK[x].size(); i++)
	{
    	int y=DSK[x][i].fi;
    	if (d[y]==d[x]+1 && f[x][y]<DSK[x][i].se && Free[y])
    	{
        	int tmp=DFS(y,min(delta,DSK[x][i].se-f[x][y]));
        	if (tmp>0)
        	{
            	f[x][y]+=tmp; f[y][x]-=tmp; return tmp;
        	}
    	}
	}
	return 0;
}
\end{lstlisting}

	\section{Mincost}
	\begin{lstlisting}[language=C++]
int calc(int x,int y){ return (x>=0) ? y : 0-y; }
 
bool findpath()
{
	for (int i=1; i<=n; i++){ trace[i]=0; d[i]=inf; } q.push(n); d[n]=0;
	while (!q.empty())
	{
    	int u=q.front(); q.pop(); inq[u]=false;
    	for (int i=0; i<DSK[u].size(); i++)
    	{
        	int v=DSK[u][i];
        	if (c[u][v]>f[u][v] && d[v]>d[u]+calc(f[u][v],cost[u][v]))
        	{
            	trace[v]=u;
                d[v]=d[u]+calc(f[u][v],cost[u][v]);
            	if (!inq[v])
            	{
           	     inq[v]=true;
                	q.push(v);
            	}
        	}
    	}
	}
	return d[t]!=inf;
}
 
void incflow()
{
	int v=t,delta=inf;
	while (v!=n)
	{
    	int u=trace[v];
    	if (f[u][v]>=0) delta=min(delta,c[u][v]-f[u][v]);
    	else delta=min(delta,0-f[u][v]);
    	v=u;
	}
	v=t;
	while (v!=n)
	{
    	int u=trace[v];
    	f[u][v]+=delta; f[v][u]-=delta;
    	v=u;
	}
}
\end{lstlisting}
	\section{HLD}
	\begin{lstlisting}[language=C++]
void DFS(int x,int pa)
{
    DD[x]=DD[pa]+1; child[x]=1; int Max=0;
    for (int i=0; i<DSK[x].size(); i++)
    {
        int y=DSK[x][i].fi;
        if (y==pa) continue;
        p[y]=x;
        d[y]=d[x]+DSK[x][i].se;
        DFS(y,x);
        child[x]+=child[y];
        if (child[y]>Max)
        {
            Max=child[y];
            tree[x]=tree[y];
        }
    }
    if (child[x]==1) tree[x]=++nTree;
}

void init()
{
    nTree=0;
    DFS(1,1);
    DD[0]=long(1e9);
    for (int i=1; i<=n; i++) if (DD[i]<DD[root[tree[i]]]) root[tree[i]]=i;
}

int LCA(int u,int v)
{
    while (tree[u]!=tree[v])
    {
        if (DD[root[tree[u]]]<DD[root[tree[v]]]) v=p[root[tree[v]]];
        else u=p[root[tree[u]]];
    }
    if (DD[u]<DD[v]) return u; else return v;
}
\end{lstlisting}
	\section{Cầu khớp}
Nút u là khớp: 
if (low[v] >= num[u]) arti[u] = arti[u] || p[u] != -1 || child[u] >= 2;\\
Cạnh u, v là cầu khi low[v] >= num[v]
	\section{Monotone chain}
	\begin{lstlisting}[language=C++]
void convex_hull (vector<pt> & a) {
  if (a.size() == 1) { // chỉ có 1 điểm
    return;
  }

  // Sort with respect to x and then y
  sort(a.begin(), a.end(), &cmp);

  pt p1 = a[0],  p2 = a.back();

  vector<pt> up, down; 
  up.push_back (p1);
  down.push_back (p1);

  for (size_t i=1; i<a.size(); ++i) { 
    // Add to the upper chain
    
    if (i==a.size()-1 || cw (p1, a[i], p2)) {
      while (up.size()>=2 && !cw (up[up.size()-2], up[up.size()-1], a[i]))
        up.pop_back();
      up.push_back (a[i]);
    }

    // Add to the lower chain
    if (i==a.size()-1 || ccw (p1, a[i], p2)) {
      while (down.size()>=2 && !ccw (down[down.size()-2], down[down.size()-1], a[i]))
        down.pop_back();
      down.push_back (a[i]);
    }
  }

  // Merge 2 chains
  a.clear();
  for (size_t i=0; i<up.size(); ++i)
    a.push_back (up[i]);
  for (size_t i=down.size()-2; i>0; --i)
    a.push_back (down[i]);
}
\end{lstlisting}
	\section{MST}
	Prim: remember to have visited array
	\section{Bignum mul}
	\begin{lstlisting}[language=C++]
string mul(string a,string b)
{
   int m=a.length(),n=b.length(),sum=0;
   string c="";
   for (int i=m+n-1; i>=0; i--)
   {
       for (int j=0; j<m; j++) if (i-j>0 && i-j<=n) sum+=(a[j]-'0')*(b[i-j-1]-'0');
       c=(char)(sum%10+'0')+c;
       sum/=10;
   }
   while (c.length()>1 && c[0]=='0') c.erase(0,1);
   return c;
}
\end{lstlisting}
	\section{Prime under 100}
	2, 3, 5, 7, 11, 13, 17, 19, 23, 29, 31, 37, 41, 43, 47, 53, 59, 61, 67, 71, 73, 79, 83, 89, 97 
	\section{Pascal triangle C(n,k)=number from line 0, column 0}
1\\
1 1\\
1 2 1\\ 
1 3 3 1\\ 
1 4 6 4 1\\ 
1 5 10 10 5 1\\ 
1 6 15 20 15 6 1\\ 
1 7 21 35 35 21 7 1\\
1 8 28 56 70 56 28 8 1\\ 
1 9 36 84 126 126 84 36 9 1\\
1 10 45 120 210 252 210 120 45 10 1
	\section{Fibo}
	0 1 1 2 3 5 8 13 21 34 55 89 144 233 377 610 987 1597 2584 4181 6765
	\section{Tips}
	\begin{enumerate}[topsep=0pt, partopsep=0pt, itemsep=0pt]
	\item Giả sử nó là số nguyên tố đi. Giả sử nó liên quan tới số nguyên tố đi.\\
	\item Giả sử nó là số có dạng \(2^n\) đi.\\
	\item Giả sử chọn tối đa là 2, 3 số gì là có đáp án đi.\\
	\item Có liên quan gì tới Fibonacci hay tam giác pascal?\\
	\item Dãy này đơn điệu không em ei? Hay tổng của 2,3 số fibonacci?\\
	\item q <= 2\\
	\item Sort lại đi, biết đâu thấy điều hay hơn?\\
	\item Chia nhỏ ra xem.\\
	\item Bỏ hết những thằng ko cần thiết ra\\
	\item Áp đại data struct nào đấy vô\\
	\item khóc\\
	\item Cầu nguyện\\
	\item Random shuffe để ac\\
	\item Xoay mảng 45 độ\\
	\end{enumerate}



\end{document}