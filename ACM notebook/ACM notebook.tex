\documentclass[A4 paper, 12pt, oneside, landscape]{article}

\usepackage[a4paper,left=0.8cm,right=0.8cm,top=1.8cm,bottom=0.7cm]{geometry}
\usepackage{listings}
\usepackage{xcolor}
\usepackage[utf8]{vietnam}
\usepackage{enumitem}
\usepackage{amsmath}
\usepackage{multicol}
\usepackage{indentfirst}
\usepackage{breqn}
\usepackage{pdflscape}

\definecolor{codegreen}{rgb}{0,0.6,0}
\definecolor{codegreener}{rgb}{0,0.7,0}
\definecolor{codepurple}{rgb}{0.58,0,0.82}
\definecolor{backcolour}{rgb}{1,1,1}
\definecolor{codeblue}{rgb}{0.1,0.457,1} 

\usepackage{fancyhdr}
\pagestyle{fancy}
\lhead{VNUHCM - University of Science, HCMUS-Senowa notebook}
\rhead{Page \thepage}
%\cfoot{Notebook}


\lstdefinestyle{mystyle}{
	language=C++,
    backgroundcolor=\color{backcolour},
    commentstyle=\color{codegreen},
    keywordstyle=\color{codeblue},
    numberstyle=\color{codegreener},
    stringstyle=\color{codepurple},
    basicstyle=\ttfamily\footnotesize,
    breakatwhitespace=false,         
    breaklines=true,                 
    captionpos=b,                    
    keepspaces=true,                 
    numbers=none,                    
    numbersep=2pt,                  
    showspaces=false,                
    showstringspaces=false,
    showtabs=false,                  
    tabsize=2
}

\lstset{style=mystyle}

\title{ACM Notebook team HCMUS-KMN}
\author{KMN}

\setlength{\columnseprule}{0.4pt}

\begin{document}

	\begin{multicols}{2}
	\tableofcontents
	\small
	
\section{Some definition}
	\lstinputlisting[language=C++]{define.cpp}

\section{Data structure}
	\subsection{Mo's algorithm}
    \[O(N * \sqrt{N} + Q * \sqrt{N}) \]
	\begin{lstlisting}[language=C++]
    S = sqrt(N);
    bool cmp(Query A, Query B) // compare 2 queries
    {
      if (A.l / S != B.l / S) {
        return A.l / S < B.l / S;
      }
      return A.r < B.r;
    }
    \end{lstlisting}
	
	\subsection{Set, map, multiset}
	Use set.lower_bound() instead of lower_bound(set.begin(), set.end()) for better performance. The same is true for map. \par 
	Set and map is sorted internally, so *s.begin() will be the smallest element and *s.rbegin() will be the largest element. \par 
	If you want to remove 1 occurrence of 1 value in the multiset, use s.erase(s.find(value)). If we use s.erase(value), the multiset will erase all occurrence of that value in the multiset.
	
	\subsection{Bitset}
	Order positions are counted from the rightmost bit, which is order position 0. \par 
	Bitset can use bit operator like normal number. \par
	If bitset is const-qualified then the return value of [] operator is bool, otherwise it is bitset::reference. \par
	set() sets all value in bitset to 1. set(pos, val) sets value at pos to val (can throw out of bound). \par
	reset() resets all value in bitset to 0. reset(pos) resets value at pos to 0 (can throw out of bound). \par 
	flip() flips all bits in bitset. flip(pos) flips bit at pos. \par 
	all() returns true if all bits are set. none() returns true if no bit is set. any() returns true if any of the bits is set. \par 
	count() returns the number of bits which are set. \par 
	test(pos) returns true if the bit at pos is set.
	
	\subsection{BIT}
	\lstinputlisting{BIT.cpp}
	
	\subsection{IT2D}
	\lstinputlisting{IT2D.cpp}
	
	\subsection{Peristent IT}
	\lstinputlisting{PersistentIT.cpp}
	
	\subsection{Li Chao Tree}
	\lstinputlisting{LiChaoTree.cpp}

\section{Graph}
	\subsection{Dinic}
	\lstinputlisting{Dinic.cpp}

	\subsection{Mincost}
	\lstinputlisting{mincost.cpp}
	
	\subsection{HLD}
	\lstinputlisting{HLD.cpp}
	
	\subsection{Tarjan}
If u is articulation: \\
if (low[v] $>=$ num[u]) arti[u] = arti[u] or p[u] $!=$ -1 or child[u] $>=$ 2; \\
If (u, v) is bridge: $low[v] >= num[v]$
	\subsection{Monotone chain}
	\lstinputlisting[language=C++]{monotone.cpp}

	\subsection{MST}
	Prim: remember to have visited array
	
	\subsection{HopcroftKarp}
	\lstinputlisting{HopcroftKarp.cpp}
	
	\subsection{Hungarian}
	\lstinputlisting{Hungarian.cpp}
	
\section{String}
	\subsection{Aho Corasick}
	\lstinputlisting{AhoCorasick.cpp}
	
	\subsection{Manacher}
	\lstinputlisting{Manacher.cpp}
	
	\subsection{Suffix Array}
	\lstinputlisting{suffixarray.cpp}
	
	\subsection{Z function}
	\lstinputlisting{zfunction.cpp}
	
	\subsection{KMP}
	\lstinputlisting{KMP.cpp}
	
	\subsection{Lexicographically minimal string rotation}
	\begin{lstlisting}
def least_rotation(S: str) -> int:
  """Booth's algorithm."""
  S += S  // Concatenate string to it self to avoid modular arithmetic
  f = [-1] * len(S)  // Failure function
  k = 0  // Least rotation of string found so far
  for j in range(1, len(S)):
    sj = S[j]
    i = f[j - k - 1]
    while i != -1 and sj != S[k + i + 1]:
      if sj < S[k + i + 1]:
        k = j - i - 1
      i = f[i]
    if sj != S[k + i + 1]:  // if sj != S[k+i+1], then i == -1
      if sj < S[k]:  # k+i+1 = k
        k = j
      f[j - k] = -1
    else:
      f[j - k] = i + 1
  return k
	\end{lstlisting}
	
	\subsection{Hash}
	\lstinputlisting{Hash.cpp}
	
	\subsection{Hash 2D}
	\begin{dmath}
	H[i][j] = H[i - 1][j] * p + H[i][j - 1] * q - H[i - 1][j - 1] * p * q + s[i][j]
	\end{dmath}
	\begin{dmath}
	Hash(a,b)(x,y) = H[x][y] - H[a - 1][y]*p^{x - a + 1} - H[x][b - 1]*q^{y - b + 1} + H[a - 1][b - 1] * p^{x - a + 1} * q^{y - b + 1}
	\end{dmath}

\section{Math}	
    \subsection{Inverse of 2x2 matrix}
    \begin{align*}
        \begin{bmatrix}
        a & b \\
        c & d \\
        \end{bmatrix} ^ {-1} = \frac{1}{ad - bc}
        \begin{bmatrix}
            d & -b \\
            -c & a \\
        \end{bmatrix}
    \end{align*}
    
    \subsection{Sum of divisors}
    If $n = p^{e_1}_1 * p^{e_2}_2 * \dots * p^{e_k}_k$, then
    \begin{align*}
        sum = \frac{p^{e_1 + 1}_1 - 1}{p_1 - 1} * \frac{p^{e_2 + 1}_2 - 1}{p_2 - 1} * \dots * \frac{p^{e_k + 1}_k - 1}{p_k - 1}
    \end{align*}
    
    \subsection{Pisano period}
    $\pi(n) \leq 6n.$ $\pi(n) = 6n$ when $n = 2 \times 5^r$ for $r > 1$. \par 
    $\pi(2) = 3, \pi(5) = 20$. \par 
    If $m$ and $n$ are coprime, $\pi(mn) = LCM(\pi(m), \pi(n))$. \par
    If $p$ is prime, $\pi(p^k)$ divides $p^{k - 1} \times \pi(p)$. It is conjectured that $\pi(p^k) = p^{k - 1} \times \pi(p)$ for $k > 1$. \par
    If $p \equiv 1 \ mod \ 10$ or $p \equiv 9 \ mod \ 10$, $\pi(p)$ is divisor of $p - 1$. \par 
    If $p \equiv 3 \ mod \ 10$ or $p \equiv 7 \ mod \ 10$, $\pi(p)$ is divisor of $2(p + 1)$.
    
    \subsection{Number Theory}
    \begin{align*}
        a + b = a \oplus b + 2 \times (a \wedge b)
    \end{align*}
    
	\subsection{Derivatives and integrals}
	\begin{align*}
        \dfrac{d}{dx}\ln{u} = \dfrac{u'}{u} &&& \dfrac{d}{dx}\dfrac{1}{u} = -\dfrac{u'}{u^2} \\
    	\dfrac{d}{dx}\sqrt u = \dfrac{u'}{2\sqrt u} \\
    	\dfrac{d}{dx}\sin x = \cos x &&& \dfrac{d}{dx}\arcsin x = \dfrac{1}{\sqrt{1-x^2}} \\ 
    	\dfrac{d}{dx}\cos x = -\sin x &&& \dfrac{d}{dx}\arccos x = -\dfrac{1}{\sqrt{1-x^2}} \\
    	\dfrac{d}{dx}\tan x = 1+\tan^2 x &&& \dfrac{d}{dx}\arctan x = \dfrac{1}{1+x^2} \\
    	\int\tan ax = -\dfrac{\ln|\cos ax|}{a} &&& \int x\sin ax = \dfrac{\sin ax-ax \cos ax}{a^2} \\
    	\int e^{-x^2} = \frac{\sqrt \pi}{2} \text{erf}(x) &&& \int xe^{ax}dx = \frac{e^{ax}}{a^2}(ax-1)
	\end{align*}

    Integration by parts:
    \[\int_a^bf(x)g(x)dx = [F(x)g(x)]_a^b-\int_a^bF(x)g'(x)dx\]	
	
	\subsection{Sum}
	\begin{align*}
    	1 + 2 + 3 + \dots + n &= \frac{n(n+1)}{2} \\
    	1^2 + 2^2 + 3^2 + \dots + n^2 &= \frac{n(2n+1)(n+1)}{6} \\
    	1^3 + 2^3 + 3^3 + \dots + n^3 &= \frac{n^2(n+1)^2}{4} \\
    	1^4 + 2^4 + 3^4 + \dots + n^4 &= \frac{n(n+1)(2n+1)(3n^2 + 3n - 1)}{30} \\
    \end{align*}
    
    \subsection{Series}
    $$e^x = 1+x+\frac{x^2}{2!}+\frac{x^3}{3!}+\dots,\,(-\infty<x<\infty)$$
    $$\ln(1+x) = x-\frac{x^2}{2}+\frac{x^3}{3}-\frac{x^4}{4}+\dots,\,(-1<x\leq1)$$
    $$\sqrt{1+x} = 1+\frac{x}{2}-\frac{x^2}{8}+\frac{2x^3}{32}-\frac{5x^4}{128}+\dots,\,(-1\leq x\leq1)$$
    $$\sin x = x-\frac{x^3}{3!}+\frac{x^5}{5!}-\frac{x^7}{7!}+\dots,\,(-\infty<x<\infty)$$
    $$\cos x = 1-\frac{x^2}{2!}+\frac{x^4}{4!}-\frac{x^6}{6!}+\dots,\,(-\infty<x<\infty)$$
    
    \subsection{Trigonometric}
    \begin{align*}
        \sin(v+w)&= \sin v\cos w+\cos v\sin w \\
        \cos(v+w)&=\cos v\cos w-\sin v\sin w\ \\
        \tan(v+w)&=\dfrac{\tan v+\tan w}{1-\tan v\tan w} \\
        \sin v+\sin w&=2\sin\dfrac{v+w}{2}\cos\dfrac{v-w}{2} \\
        \cos v+\cos w&=2\cos\dfrac{v+w}{2}\cos\dfrac{v-w}{2} 
    \end{align*}
    \begin{align*}
        a\cos x+b\sin x&=r\cos(x-\phi)\\
        a\sin x+b\cos x&=r\sin(x+\phi)\\
    \end{align*}
    where $r=\sqrt{a^2+b^2}, \phi=\operatorname{atan2}(b,a)$.

    \subsection{Inverse}
    Calculate inverse for [1, m - 1]. O(m). m is prime
    \begin{lstlisting}
inv[1] = 1;
for(int i = 2; i < m; ++i)
    inv[i] = m - (m/i) * inv[m%i] % m;
    \end{lstlisting}
    
	\subsection{Gaussian elimination}
	\lstinputlisting{Gaussian.cpp}	    

	\subsection{Geometry}
	\lstinputlisting{geometry.cpp}
	
	\subsection{Discrete logarithm}
	\lstinputlisting{discreteLog.cpp}
	
	\subsection{Miller Rabin}
	\lstinputlisting{MillerRabin.cpp}
	
	\subsection{Miller Rabin Deterministic}
	\lstinputlisting{MillerRabinRR.cpp}

	\subsection{Chinese Remainer}
	\lstinputlisting{Chinese.cpp}
	
	\subsection{Extended Euclid}
	\lstinputlisting{ExtendedEuclid.cpp}

	\subsection{FFT}
	\lstinputlisting{fft_handmade.cpp}

	\subsection{PollardRho}
	\lstinputlisting{PollardRho.cpp}

\section{Theorem}
	\subsection{Fermat's little theorem}
	If $p$ is a prime number, then for any number $a$, \(a^p - a\) is an integer multiple of $p$ \\
	\[a ^ p \equiv a \ (mod \ p)\]  \\
	If $a$ is not divisible by $p$ \\
	\[a ^ {p - 1} \equiv 1 \ (mod \ p)\] 
	
	\subsection{Euler's theorem}
	If $a$ and $n$ are coprime, then 
	\[a ^ {\phi(n)} \equiv 1 \ (mod \ n) \]
	
	\subsection{Euler's totient function}
	The number of coprime $\leq n$ \\
	\[\phi(n) = n \prod (1 - \frac{1}{p}) \] \\
	With $p$ is the prime divided by $n$
	
	\subsection{Goldbach's conjecture}
	Every even number greater than 2 is the sum of 2 primes. \(\leq 4 * 10^{18}\)
	
	\subsection{Dirichlet}
	Given $n$ holes and $n + 1$ pigeons to distribute evenly, then at least $1$ hole must have $2$ pigeons
	
	\subsection{Pythagorean triple}
	\[a = m^2 - n^2, \ b = 2mn, \ c = m^2 + n^2 \] \\
	Where $m$ and $n$ are positive integer with $m > n$, and with $m$ and $n$ are coprime and not both odd. When both m and n are odd, then a, b, and c will be even, and the triple will not be primitive; however, dividing a, b, and c by 2 will yield a primitive triple when m and n are coprime and both odd.
	
	Despite generating all primitive triples, Euclid's formula does not produce all triples—for example, (9, 12, 15) cannot be generated using integer m and n. This can be remedied by inserting an additional parameter k to the formula. The following will generate all Pythagorean triples uniquely:

    \[a=k (m^{2}-n^{2}),\ \,b=k (2mn),\ \,c=k (m^{2}+n^{2}) \]
    Where $m$, $n$, and $k$ are positive integers with $m > n$, and with $m$ and $n$ coprime and not both odd.
	
	\subsection{Legendre's formula}
	Factor $n!$
	\[v_p(n!) = \sum_{i = 1}^{\infty} \left \lfloor \frac{n}{p ^ i} \right \rfloor \]
	With $p$ is prime
	
	\subsection{Stirling's approximation}
	\[n! \approx \sqrt{2\pi n} * (\frac{n}{e}) ^ n\]
	
	\subsection{Wilson's theorem}
	$n > 1$ is a prime if and only if 
	\[ (n - 1)! \ mod \ n \equiv -1 \ mod \ n \]

\section{Other}
    \subsection{Matrix}
    \lstinputlisting{matrix.cpp}

	\subsection{Bignum mul}
	\lstinputlisting{bignum.cpp}
	
	\subsection{Random}
	\lstinputlisting{random.cpp}
	
	\subsection{Builtin bit function}
	\lstinputlisting{bitfunction.cpp}
	
	\subsection{Pythagorean triples}
	c under 100 there are 16 triples:
	(3, 4, 5)	(5, 12, 13)	(8, 15, 17)	(7, 24, 25)
(20, 21, 29)	(12, 35, 37)	(9, 40, 41)	(28, 45, 53)
(11, 60, 61)	(16, 63, 65)	(33, 56, 65)	(48, 55, 73)
(13, 84, 85)	(36, 77, 85)	(39, 80, 89)	(65, 72, 97) \\

	$100 \leq c \leq 300$:
	(20, 99, 101)	(60, 91, 109)	(15, 112, 113)	(44, 117, 125)
(88, 105, 137)	(17, 144, 145)	(24, 143, 145)	(51, 140, 149)
(85, 132, 157)	(119, 120, 169)	(52, 165, 173)	(19, 180, 181)
(57, 176, 185)	(104, 153, 185)	(95, 168, 193)	(28, 195, 197)
(84, 187, 205)	(133, 156, 205)	(21, 220, 221)	(140, 171, 221)
(60, 221, 229)	(105, 208, 233)	(120, 209, 241)	(32, 255, 257)
(23, 264, 265)	(96, 247, 265)	(69, 260, 269)	(115, 252, 277)
(160, 231, 281)	(161, 240, 289)	(68, 285, 293)

	\subsection{Sieve}
	\lstinputlisting{Sieve.cpp}

    \subsection{Factorial mod}
    \begin{lstlisting}
int factmod(int n, int p) {
    vector<int> f(p);
    f[0] = 1;
    for (int i = 1; i < p; i++)
        f[i] = f[i-1] * i % p;

    int res = 1;
    while (n > 1) {
        if ((n/p) % 2)
            res = p - res;
        res = res * f[n%p] % p;
        n /= p;
    }
    return res; 
}
    \end{lstlisting}

	\subsection{Catalan}
	\[\frac{(2n)!}{(n + 1)!n!} = \prod_{k = 2}^n \frac{n + k}{k} \]
	The Catalan number Cn is the solution for
	\begin{itemize}
	    \item Number of correct bracket sequence consisting of $n$ opening and n closing brackets.
	    \item The number of rooted full binary trees with $n+1$ leaves (vertices are not numbered). A rooted binary tree is full if every vertex has either two children or no children.
	    \item The number of ways to completely parenthesize $n+1$ factors.
	    \item The number of triangulations of a convex polygon with $n+2$ sides (i.e. the number of partitions of polygon into disjoint triangles by using the diagonals).
	    \item The number of ways to connect the $2n$ points on a circle to form n disjoint chords.
	    \item The number of non-isomorphic full binary trees with $n$ internal nodes (i.e. nodes having at least one son).
	    \item The number of monotonic lattice paths from point (0,0) to point (n,n) in a square lattice of size $n \times n$, which do not pass above the main diagonal (i.e. connecting (0,0) to (n,n)).
	    \item Number of permutations of length $n$ that can be stack sorted (i.e. it can be shown that the rearrangement is stack sorted if and only if there is no such index $i<j<k$, such that $a_k<a_i<a_j$ ).
	    \item The number of non-crossing partitions of a set of $n$ elements.
        \item The number of ways to cover the ladder $1...n$ using $n$ rectangles (The ladder consists of $n$ columns, where $i^{th}$ column has a height $i$).
	\end{itemize}

	\subsection{Prime under 100}
	2, 3, 5, 7, 11, 13, 17, 19, 23, 29, 31, 37, 41, 43, 47, 53, 59, 61, 67, 71, 73, 79, 83, 89, 97 
	
	\subsection{Pascal triangle}
C(n,k)=number from line 0, column 0\\
1\\
1 1\\
1 2 1\\ 
1 3 3 1\\ 
1 4 6 4 1\\ 
1 5 10 10 5 1\\ 
1 6 15 20 15 6 1\\ 
1 7 21 35 35 21 7 1\\
1 8 28 56 70 56 28 8 1\\ 
1 9 36 84 126 126 84 36 9 1\\
1 10 45 120 210 252 210 120 45 10 1
	\subsection{Fibo}
	0 1 1 2 3 5 8 13 21 34 55 89 144 233 377 610 987 1597 2584 4181 6765
	\section{Tips}
	\begin{itemize}[topsep=0pt, partopsep=0pt, itemsep=0pt]
	\item 5' debug vẫn nhanh hơn 20' penalty
	\item Giả sử nó là số nguyên tố đi. Giả sử nó liên quan tới số nguyên tố đi.
	\item Giả sử nó là số có dạng $2^n$ đi.
	\item Giả sử chọn tối đa là 2, 3 số gì là có đáp án đi.
	\item Có liên quan gì tới Fibonacci hay tam giác pascal?
	\item Dãy này đơn điệu không em ei? Hay tổng của 2, 3 số fibonacci?
	\item Chia nhỏ ra xem.
	\item Random shuffe để AC
	\item Xoay mảng 45 độ (thường liên quan đến Manhattan)
	\item Tạo đỉnh ảo cho đồ thị (vd như Kruskal)
	\item Tìm t thỏa điều kiện nào đó thì chặt
	\item Merge set thì phải merge từ set nhỏ sang lớn ko thì TLE
	\item Xử lý ma trận cũng giống xử lý số bình thường, các phép nhân chia mod đều như cũ
	\item Làm luồng nhớ push cung ngược
	\item Nếu xử lý liên quan đến bit thì có thể nó liên quan tới Trie
	\item Số nguyên tố thường có dạng $6k \pm 1$ trừ 2 và 3
	\end{itemize}

	\end{multicols}

    \begin{figure}
        \centering
        \includegraphics{renai.png}
    \end{figure}

\end{document}