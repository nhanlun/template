\documentclass[A4 paper, 12pt, oneside]{article}

\usepackage[a4paper,left=0.8cm,right=0.8cm,top=2cm,bottom=2cm]{geometry}
\usepackage{listings}
\usepackage{xcolor}
\usepackage[utf8]{inputenc}
\usepackage{enumitem}
\usepackage{amsmath}
\usepackage{multicol}
\usepackage{indentfirst}
\usepackage{breqn}
\usepackage{pdflscape}

%\usepackage{mathtools}
%\DeclarePairedDelimiter{\ceil}{\lceil}{\rceil}
%\DeclarePairedDelimiter{\floor}{\lfloor}{\rfloor}

\definecolor{codegreen}{rgb}{0,0.6,0}
\definecolor{codegreener}{rgb}{0,0.7,0}
\definecolor{codepurple}{rgb}{0.58,0,0.82}
\definecolor{backcolour}{rgb}{1,1,1}
\definecolor{codeblue}{rgb}{0.1,0.457,1} 

\usepackage{fancyhdr}
\pagestyle{fancy}
\lhead{HCMUS-KMN}
\rhead{Page \thepage}
%\cfoot{Notebook}


\lstdefinestyle{mystyle}{
	language=C++,
    backgroundcolor=\color{backcolour},
    commentstyle=\color{codegreen},
    keywordstyle=\color{codeblue},
    numberstyle=\color{codegreener},
    stringstyle=\color{codepurple},
    basicstyle=\ttfamily\footnotesize,
    breakatwhitespace=false,         
    breaklines=true,                 
    captionpos=b,                    
    keepspaces=true,                 
    numbers=none,                    
    numbersep=2pt,                  
    showspaces=false,                
    showstringspaces=false,
    showtabs=false,                  
    tabsize=2
}

\lstset{style=mystyle}

\title{ACM Notebook team HCMUS-KMN}
\author{KMN}

\setlength{\columnseprule}{0.4pt}

\begin{document}

	\begin{landscape}
	\begin{multicols}{2}
	\tableofcontents
	\small
	
\section{Some definition}
	\lstinputlisting[language=C++]{define.cpp}

\section{Data structure}
	\subsection{Mo's algorithm}
\[O(N * \sqrt{N} + Q * \sqrt{N}) \]
	\begin{lstlisting}[language=C++]
S = sqrt(N);
bool cmp(Query A, Query B) // compare 2 queries
{
  if (A.l / S != B.l / S) {
    return A.l / S < B.l / S;
  }
  return A.r < B.r;
}
\end{lstlisting}
	
	\subsection{Set and map}
	Use set.lower_bound() instead of lower_bound(set.begin(), set.end()) for better performance \\
	The same is true for map
	
	\subsection{BIT}
	\lstinputlisting{BIT.cpp}
	
	\subsection{IT2D}
	\lstinputlisting{IT2D.cpp}

\section{Graph}
	\subsection{Dinic}
	\lstinputlisting{Dinic.cpp}

	\subsection{Mincost}
	\lstinputlisting{mincost.cpp}
	
	\subsection{HLD}
	\lstinputlisting{HLD.cpp}
	
	\subsection{Cầu khớp}
Nút u là khớp: 
if (low[v] >= num[u]) arti[u] = arti[u] || p[u] != -1 || child[u] >= 2;\\
Cạnh u, v là cầu khi low[v] >= num[v]
	\subsection{Monotone chain}
	\lstinputlisting[language=C++]{monotone.cpp}

	\subsection{MST}
	Prim: remember to have visited array
	
\section{String}
	\subsection{Aho Corasick}
	\lstinputlisting{AhoCorasick.cpp}
	
	\subsection{Manacher}
	\lstinputlisting{Manacher.cpp}
	
	\subsection{Suffix Array}
	\lstinputlisting{suffixarray.cpp}
	
	\subsection{Z function}
	\lstinputlisting{zfunction.cpp}
	
	\subsection{KMP}
	\lstinputlisting{KMP.cpp}
	
	\subsection{Hash 2D}
	\begin{dmath}
	H[i][j] = H[i - 1][j] * p + H[i][j - 1] * q - H[i - 1][j - 1] * p * q + s[i][j]
	\end{dmath}
	\begin{dmath}
	Hash(a,b)(x,y) = H[x][y] - H[a - 1][y]*p^{x - a + 1} - H[x][b - 1]*q^{y - b + 1} + H[a - 1][b - 1] * p^{x - a + 1} * q^{y - b + 1}
	\end{dmath}

\section{Math}	
	\subsection{Derivatives and integrals}
	\begin{align*}
        \dfrac{d}{dx}\ln{u} = \dfrac{u'}{u} &&& \dfrac{d}{dx}\dfrac{1}{u} = -\dfrac{u'}{u^2} \\
    	\dfrac{d}{dx}\sqrt u = \dfrac{u'}{2\sqrt u} \\
    	\dfrac{d}{dx}\sin x = \cos x &&& \dfrac{d}{dx}\arcsin x = \dfrac{1}{\sqrt{1-x^2}} \\ 
    	\dfrac{d}{dx}\cos x = -\sin x &&& \dfrac{d}{dx}\arccos x = -\dfrac{1}{\sqrt{1-x^2}} \\
    	\dfrac{d}{dx}\tan x = 1+\tan^2 x &&& \dfrac{d}{dx}\arctan x = \dfrac{1}{1+x^2} \\
    	\int\tan ax = -\dfrac{\ln|\cos ax|}{a} &&& \int x\sin ax = \dfrac{\sin ax-ax \cos ax}{a^2} \\
    	\int e^{-x^2} = \frac{\sqrt \pi}{2} \text{erf}(x) &&& \int xe^{ax}dx = \frac{e^{ax}}{a^2}(ax-1)
	\end{align*}

    Integration by parts:
    \[\int_a^bf(x)g(x)dx = [F(x)g(x)]_a^b-\int_a^bF(x)g'(x)dx\]	
	
	\subsection{Sum}
	\begin{align*}
    	1 + 2 + 3 + \dots + n &= \frac{n(n+1)}{2} \\
    	1^2 + 2^2 + 3^2 + \dots + n^2 &= \frac{n(2n+1)(n+1)}{6} \\
    	1^3 + 2^3 + 3^3 + \dots + n^3 &= \frac{n^2(n+1)^2}{4} \\
    	1^4 + 2^4 + 3^4 + \dots + n^4 &= \frac{n(n+1)(2n+1)(3n^2 + 3n - 1)}{30} \\
    \end{align*}
    
    \subsection{Series}
    $$e^x = 1+x+\frac{x^2}{2!}+\frac{x^3}{3!}+\dots,\,(-\infty<x<\infty)$$
    $$\ln(1+x) = x-\frac{x^2}{2}+\frac{x^3}{3}-\frac{x^4}{4}+\dots,\,(-1<x\leq1)$$
    $$\sqrt{1+x} = 1+\frac{x}{2}-\frac{x^2}{8}+\frac{2x^3}{32}-\frac{5x^4}{128}+\dots,\,(-1\leq x\leq1)$$
    $$\sin x = x-\frac{x^3}{3!}+\frac{x^5}{5!}-\frac{x^7}{7!}+\dots,\,(-\infty<x<\infty)$$
    $$\cos x = 1-\frac{x^2}{2!}+\frac{x^4}{4!}-\frac{x^6}{6!}+\dots,\,(-\infty<x<\infty)$$
    
    \subsection{Trigonometric}
    \begin{align*}
        \sin(v+w)&= \sin v\cos w+\cos v\sin w \\
        \cos(v+w)&=\cos v\cos w-\sin v\sin w\ \\
        \tan(v+w)&=\dfrac{\tan v+\tan w}{1-\tan v\tan w} \\
        \sin v+\sin w&=2\sin\dfrac{v+w}{2}\cos\dfrac{v-w}{2} \\
        \cos v+\cos w&=2\cos\dfrac{v+w}{2}\cos\dfrac{v-w}{2} 
    \end{align*}
    \begin{align*}
        a\cos x+b\sin x&=r\cos(x-\phi)\\
        a\sin x+b\cos x&=r\sin(x+\phi)\\
    \end{align*}
    where $r=\sqrt{a^2+b^2}, \phi=\operatorname{atan2}(b,a)$.

	\subsection{Gaussian elimination}
	\lstinputlisting{Gaussian.cpp}	    

	\subsection{Geometry}
	\lstinputlisting{geometry.cpp}
	
	\subsection{Miller Rabin}
	\lstinputlisting{MillerRabin.cpp}

	\subsection{Chinese Remainer}
	\lstinputlisting{Chinese.cpp}
	
	\subsection{Extended Euclid}
	\lstinputlisting{ExtendedEuclid.cpp}

	\subsection{FFT}
	\lstinputlisting{fft_handmade.cpp}
	
	\subsection{Hungarian}
	\lstinputlisting{Hungarian.cpp}

	\subsection{PollardRho}
	\lstinputlisting{PollardRho.cpp}

\section{Theorem}
	\subsection{Fermat's little theorem}
	If $p$ is a prime number, then for any number $a$, \(a^p - a\) is an integer multiple of $p$ \\
	\[a ^ p \equiv a \ (mod \ p)\]  \\
	If $a$ is not divisible by $p$ \\
	\[a ^ {p - 1} \equiv 1 \ (mod \ p)\] 
	
	\subsection{Euler's totient function}
	The number of coprime $\leq n$ \\
	\[\phi(n) = n \prod (1 - \frac{1}{p}) \] \\
	With $p$ is the prime divided by $n$
	
	\subsection{Dirichlet}
	Given $n$ holes and $n + 1$ pigeons to distribute evenly, then at least $1$ hole must have $2$ pigeons
	
	\subsection{Pythagorean triple}
	\[a = m^2 - n^2, \ b = 2mn, \ c = m^2 + n^2 \] \\
	where $m$ and $n$ are positive integer with $m > n$, and with $m$ and $n$ are coprime and not both odd.
	
	\subsection{Legendre's formula}
	Factor $n!$
	\[v_p(n!) = \sum_{i = 1}^{\infty} \left \lfloor \frac{n}{p ^ i} \right \rfloor \]
	With $p$ is prime

\section{Other}
	\subsection{Bignum mul}
	\lstinputlisting{bignum.cpp}
	
	\subsection{Random}
	\lstinputlisting{random.cpp}
	
	\subsection{Builtin bit function}
	\lstinputlisting{bitfunction.cpp}
	
	\subsection{Pythagorean triples}
	c under 100 there are 16 triples:
	(3, 4, 5)	(5, 12, 13)	(8, 15, 17)	(7, 24, 25)
(20, 21, 29)	(12, 35, 37)	(9, 40, 41)	(28, 45, 53)
(11, 60, 61)	(16, 63, 65)	(33, 56, 65)	(48, 55, 73)
(13, 84, 85)	(36, 77, 85)	(39, 80, 89)	(65, 72, 97) \\

	$100 \leq c \leq 300$:
	(20, 99, 101)	(60, 91, 109)	(15, 112, 113)	(44, 117, 125)
(88, 105, 137)	(17, 144, 145)	(24, 143, 145)	(51, 140, 149)
(85, 132, 157)	(119, 120, 169)	(52, 165, 173)	(19, 180, 181)
(57, 176, 185)	(104, 153, 185)	(95, 168, 193)	(28, 195, 197)
(84, 187, 205)	(133, 156, 205)	(21, 220, 221)	(140, 171, 221)
(60, 221, 229)	(105, 208, 233)	(120, 209, 241)	(32, 255, 257)
(23, 264, 265)	(96, 247, 265)	(69, 260, 269)	(115, 252, 277)
(160, 231, 281)	(161, 240, 289)	(68, 285, 293)

	\subsection{Sieve}
	for (int j = i; j * i <= lim; ++j) notPrime[j * i] = true

	\subsection{Catalan}
	\[\frac{(2n)!}{(n + 1)!n!} = \prod_{k = 2}^n \frac{n + k}{k} \]

	\subsection{Prime under 100}
	2, 3, 5, 7, 11, 13, 17, 19, 23, 29, 31, 37, 41, 43, 47, 53, 59, 61, 67, 71, 73, 79, 83, 89, 97 
	\subsection{Pascal triangle}
C(n,k)=number from line 0, column 0\\
1\\
1 1\\
1 2 1\\ 
1 3 3 1\\ 
1 4 6 4 1\\ 
1 5 10 10 5 1\\ 
1 6 15 20 15 6 1\\ 
1 7 21 35 35 21 7 1\\
1 8 28 56 70 56 28 8 1\\ 
1 9 36 84 126 126 84 36 9 1\\
1 10 45 120 210 252 210 120 45 10 1
	\subsection{Fibo}
	0 1 1 2 3 5 8 13 21 34 55 89 144 233 377 610 987 1597 2584 4181 6765
	\section{Tips}
	\begin{itemize}[topsep=0pt, partopsep=0pt, itemsep=0pt]
	\item Test kĩ trước khi nộp. Code nhìn đúng chưa chắc đúng đâu
	\item Test conner case
	\item Có overflow ko?
	\item Đọc kĩ mô tả test
	\item Giả sử nó là số nguyên tố đi. Giả sử nó liên quan tới số nguyên tố đi.\\
	\item Giả sử nó là số có dạng \(2^n\) đi.\\
	\item Giả sử chọn tối đa là 2, 3 số gì là có đáp án đi.\\
	\item Có liên quan gì tới Fibonacci hay tam giác pascal?\\
	\item Dãy này đơn điệu không em ei? Hay tổng của 2,3 số fibonacci?\\
	\item \(q \leq 2\)\\
	\item Sort lại đi, biết đâu thấy điều hay hơn?\\
	\item Chia nhỏ ra xem.\\
	\item Bỏ hết những thằng ko cần thiết ra\\
	\item Áp đại data struct nào đấy vô\\
	\item Random shuffe để AC\\
	\item Xoay mảng 45 độ\\
	\end{itemize}

	\end{multicols}

	\centering
	\Huge
	\textcolor{blue}{\textbf{Keep Smilling}} \\
	\textcolor{blue}{Gotta solve them all}
	\end{landscape}
\end{document}